\chapter{Ilustrativní příklad} 
\label{ilustrativni_priklad}
Ilustrativní příklad byl řešen pomocí programu Agros2D, který využívá k řešení problémů fyzikálních polí adaptivní metodu konečných elementů vyššího řádu přesnosti($hp$-FEM). Pro psaní algoritmů pro danný problém se používá v aplikaci Agros2D jazyk Python.
\section{Popis metody řešení - METODA KONEČNÝCH ELEMENTŮ}
Analytické řešení fyzikálních problémů už od začátku vývoje počítačů slouží spíše jen pro pochopení principu jednotlivých fyzikálních zákonů a disciplín pro studenty. Tato řešení vyžadovala idealizace, zjednodušní problémů, nemluvě o tom, že některé problémy nebylo téměř možné analyticky řešit a pokud řešení bylo možné, tak velmi nepřesné. Proto se začaly se vznikem počítačů vyvíjet i numerické metody, které využívaly výpočetní techniky pro řešení těchto problémů. V dnešní době je numerických metod nepřeberné množství(Monte Carlo, momentová metoda, metoda konečných objemů, atd.). Pro řešení elektromagnetického a teplotního pole se nejvíce v minulosti využívala metoda konečných diferencí, v dnešní době je tato metoda již zcela ve všech aplikacích nahrazena metodou konečných elementů. Obě tyto metody mají společné po, že řeší parciální diferenciální rovnice.

Metoda konečných elementů vychází z již dříve odvozených metod a to z Ritzovy a Galerkinovy metody. Metoda hledá aproximované řešení parciálních diferenciálních rovnic(v některých případech integrálních rovnic).
Princip metody konečných elementů(dále jen FEM) spočívá v tom, že oblast ve které se problém řeší, se diskretizuje. To lze taktéž popsat jako rozdělení spojité oblasti na množinu jednotlivých podoblastí, jak již název metody naznačuje, počet těchto podoblastí je konečný a mají tedy konečné rozměry. Základní požadavek na diskretizaci je ten, že se žádné elementy(podoblasti) nesmějí překrývat. Při diskretizaci pomocí trojůhelníkové sítě, tvoří vrcholy jednotlivých trojůhelníků takzvané uzly, ty musejí být soumístné s vrcholy trojůhelníků sousedních. Je vhodné volit menší elementy v místech, kde jsou očekáváný vetší změny pole. Jelikož FEM je metoda, která se používá pro řešení problémů s danými okrajovými podmínkami, tak je nutné znát i okrajové podmínky úlohy. Ty mohou být dvojího typu, Dirichletova podmínka, kde má hranice oblasti jasně danou hodnotu, nebo Neumannova nulová podmínka, kde je derivace podle normály nulová(ve specifických případech může být i nenulová, to je nehomogenní Neumannova okrajová podmínka). Dále FEM, díky rozdělené oblasti, aproximuje neznámé, hledané řešení uvnitř každé podoblasti jednouduchou funkcí(například lineární), tím překonává obě metody, ze kterých vychází. V této funkci je potřeba určit koeficienty, to je řešeno soustavou rovnic pro jednotlivé uzly. Tím se dostane vyjádření aproximující funkce v elementu v závislosti na hodnotách v uzlových bodech. Dále je potřeba odvodit rovnice pro element, funkcionál\footnote[1]{Funkcionál je matematický operátor, který zobrazuje prostor funkcí(vektorový prostor) do množiny komplexních čísel(skalárů)} celé oblasti je roven součtu funkcionálů všech podoblastí. Protože potřebujeme tento funkcionál minimalizovat, musíme nejprve vyjýdřit jeho derivace podle hodnot hledané funkce v uzlech. Z totoho dostaneme soustavu rovnic. Pro řešení je nutné zadat Dirichletovu okrajovou podmínku(abychom ze soustavy rovnic byly schopni získat řešení).
