\chapter{Úvod} \label{uvod}
Tlumivky jsou hojně používané prvky převážně v silnoproudé elektrotechnice a v elektrických pohonech. Vyskytují se ve velké škále výkonů, od jednotek kVAr do stovek MVAr. 
\section{Tlumivky obecně}
Již z fyzického uspořádání tlumivky je jasné, že ve své podstatě je tlumivka cívkou. Z tohoto důvodu se jako cívka chová, otáčí fázi mezi napětím a proudem o +90$^\circ$ po imaginární ose, má vlastní indukčnost, parazitní odpor, počet závitů, atd. Z teorie elektromagnetického pole je známo, že v okolí cívek se tvoří určité pole (záleží na druhu procházejícího signálu cívkou). Jelikož se v silnoproudé elektrotechnice používá signál harmonický o frekvenci 50 HZ, bude magnetické pole tlumivek taktéž harmonické. 
\section{Rozdělení tlumivek}
Tlumivky se dělí na tlumivky vzduchové a na tlumivky s magnetickým obvodem. V mé práci se budu zabývat pouze tlumivkami vzduchovými, ale z důvodu přiblížení problematiky v krátkosti přiblížím funkci obou druhů tlumivek.
\subsection{Tlumivky s magnetickým obvodem}
\subsection{Vzduchové tlumivky}
\section{Použití tlumivek}
Jak již bylo zmíněno, tak se tlumivky používají převážně v silnoproudé elektrotechnice a elektrických pohonech. Nejčastější použití je pro kompenzaci účiníku cos $\varphi$ . To je proces, kdy se posouvá fázi proudu oproti napětí. Celkový výkon se skládá z výkonu činného a jalového. Jalový výkon se snažíme kompenzovat na co možná nejmenší hodnotu, aby celkový výkon nebyl moc velký a nepřetěžoval zdroj napětí. Kompenzace účiníku pomocí tlumivky se používá při kapacitním nebo odporově-kapacitním charakteru zátěže. Další aplikace tlumivek v elektrotechnice jsou pro omezení nadproudů a zkratových proudů, ke spouštění velkých třífázových synchronních i asynchronních motorů, pro regulaci výkonu v síti, k odstranění vyšších harmonických složek napětí a různá další použití tlumivek.
\section{Základní zapojení tlumivek} Tlumivky, ať s magnetickým obvodem nebo vzduchové mají dvě základní zapojení a to paralelní a sériové. Sériové zapojení spočívá v připojení tlumivky mezi dva uzly větve elektrického obvodu. Toto zapojení slouží pro omezení proudů v obvodu. Paralelní zapojení spočívá  v připojení tlumivky mezi uzel, který se nachází na větvi elektrického obvodu a zemí. Tento druh zapojení se používá pro kompenzaci účiníku. Kromě těchto dvou zapojení lze tlumivky zapojovat ještě v kombinaci s kondenzátorovými bateriemi.