\chapter{Úvod} \label{uvod}
Tlumivky jsou hojně používané prvky převážně v silnoproudé elektrotechnice a v elektrických pohonech. Vyskytují se ve velké škále výkonů, od jednotek kVAr do stovek MVAr. 

\section{Tlumivky obecně}
Již z fyzického uspořádání tlumivky je jasné, že ve své podstatě je tlumivka cívkou. Z tohoto důvodu se jako cívka chová, otáčí fázi mezi napětím a proudem o +90$^\circ$ , má vlastní indukčnost, parazitní odpor, počet závitů, atd. Z teorie elektromagnetického pole je známo, že v okolí cívek se tvoří elektromagnetické pole (záleží na druhu procházejícího signálu cívkou). Jelikož se v silnoproudé elektrotechnice používá signál harmonický o frekvenci 50 HZ, bude elektromagnetické pole v okolí tlumivek taktéž harmonické. 

\section{Rozdělení tlumivek}
Tlumivky se dělí na tlumivky vzduchové a na tlumivky s magnetickým obvodem. V mé práci se budu zabývat pouze tlumivkami vzduchovými, ale z důvodu přiblížení problematiky v krátkosti přiblížím funkci obou druhů tlumivek.

\subsection{Vzduchové tlumivky}
To jsou tlumivky bez feromagnetického obvodu. Vzduch je prostředí s lineární magnetizační charakteristikou, to je charakteristika mezi elektrickým proudem a magnetickým tokem, z čehož plyne, z definice vlastní indukčnosti, že vlastní indukčnost dané tlumivky bude konstantní. Tento druh tlumivky lze použít pro všechny aplikace tlumivky, kromě aplikací, kde se funkčně využívá přesycování magnetického obvodu. Lineární magnetizační charakteristika je velkou výhodou vzduchové tlumivky oproti tlumivce s magnetickým obvodem, toto si ale vybírá svou daň velkými rozměry tlumivky, velkým počtem proudovodičů cívky, z té plyne i velká hmotnost pramenící v velkého množství mědi. Největší nevýhodou tohoto druhu tlumivky je ale silné elektromagnetické pole, to zasahuje do konstrukčních částí tlumivky, ale také do elektricky vodivých předmětů v dosahu tohoto pole, tím se do materiálů indukují vířivé proudy a vznikají s tím spojené ztráty. Při používání vzduchové tlumivky je zapotřebí daleko více volného prostoru v okolí této tlumivky, než v případě tlumivky s magnetickým obvodem.   

\subsection{Tlumivky s magnetickým obvodem}
Tento druh tlumivek je proveden tak, že cívky jsou umístěny na feromagnetických obvodech. V těchto obvodech se uzavírá většina vybuzeného magnetického toku, magnetický tok mimo tento obvod se nazývá rozptylový magnetický tok, který je ve většině případů zanedbatelný(maximálně několik procent celkového magnetického toku). Feromagnetické obvody jsou poskládány ze za studena válcovaných orientovaných plechů, to se provádí kvůli snížení ztrát způsobených frekvencí. Jde o takzvané ztráty v železe nebo se také nazývají ztráty v magnetickém obvodu. V podstatě se na nich podílí skinefekt a vířivé proudy. S rostoucí frekvencí se musí snižovat tloušťka plechů. Podobně jako u transformátorů lze mít magnetický obvod jak v jádrovém, tak v plášťovém provedení. Ve většině případů, ve kterých se používá magnetický obvod v jádrovém provedení, je jádro rozděleno vzduchovými mezerami.

\section{Použití tlumivek}
Jak již bylo zmíněno, tlumivky se používají převážně v silnoproudé elektrotechnice a elektrických pohonech. Nejčastější použití je pro kompenzaci účiníku cos $\varphi$ . To je proces, kdy se posouvá fázi proudu oproti napětí. Celkový výkon se skládá z výkonu činného a jalového. Jalový výkon se snažíme kompenzovat na co možná nejmenší hodnotu, aby celkový výkon nebyl moc velký a nepřetěžoval zdroj napětí. Kompenzace účiníku pomocí tlumivky se používá při kapacitním nebo odporově-kapacitním charakteru zátěže. Další aplikace tlumivek v elektrotechnice jsou pro omezení nadproudů a zkratových proudů, ke spouštění velkých třífázových synchronních i asynchronních motorů, pro regulaci výkonu v síti, k odstranění vyšších harmonických složek napětí.

\section{Základní zapojení tlumivek} Tlumivky, ať s magnetickým obvodem nebo vzduchové mají dvě základní zapojení a to paralelní a sériové. Sériové zapojení spočívá v připojení tlumivky mezi dva uzly elektrického obvodu. Toto zapojení slouží pro omezení proudů v obvodu. Paralelní zapojení spočívá  v připojení tlumivky mezi uzel, který se nachází na větvi elektrického obvodu a zemí. Tento druh zapojení se používá pro kompenzaci účiníku. Kromě těchto dvou zapojení lze tlumivky zapojovat ještě v kombinaci s kondenzátorovými bateriemi.