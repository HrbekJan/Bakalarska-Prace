\chapter{Matematický model} 
\label{matematicky_model}
V této kapitole je uveden a vysvětlen obecný matematický model vzduchových tlumivek protékaných střídavým harmonickým proudem. Tento model popisuje slabě sdruženou úlohu, tedy úlohu řešící dvě fyzikální pole a to pole magnetické a pole teplotní. Proto je tato kapitola rozdělena do dvou částí, pro magnetické pole a pro teplotní pole.

\section{Model magnetického pole}
Model magnetického pole je odvozen z Maxwellových rovnic pro stacionární pole a následnou úpravou je převeden pro použití v nestacionárním harmonickém magnetické poli.

	Zákon celkového proudu (zobecněný Amperův zákon) neboli první Maxwellova rovnice v diferenciálním tvaru ve znění:
	
\begin{equation}
	\rot \vec{H} = \vec{J} + \frac{\partial \vec{D}}{\partial t}
	\label{amperuv_zakon}
\end{equation}

kde \vec{H} je vektor intenzity magnetického pole, \vec{J} je vektor proudové hustoty a $\frac{\partial \vec{D}}{\partial t}$ je hustota posuvného proudu. Hustotu posuvného proudu lze zanedbat, z důvodu zanedbatelné velikosti $\varepsilon$ oproti $\gamma$. Protože $\vec{J} =\gamma \vec{E} $ a $\vec{D} =\varepsilon \vec{E} $, lze předpokládat, že pokud bude $\gamma$ daleko větší než $\varepsilon$, lze člen $\frac{\partial \vec{D}}{\partial t}$ zanedbat. Ve skutečnosti je opravdu $\gamma$ daleko větší než $\varepsilon$ a proto bude člen s parciální derivací zanedbán.
Po zanedbání posuvného proudu rovnice vypadá takto:

\begin{equation}
	\rot \vec{H} = \vec{J}
	\label{vec_H=vec_J}
\end{equation}

Dalším krokem je dosazení za intenzitu magnetického pole \vec{H} ze vztahu mezi \vec{H} a magnetickou indukcí \vec{B} (\ref{permeabilita}), tím ze vztahu (\ref{vec_H=vec_J}) vznikne vztah ve tvaru:

\begin{equation}
	\rot \frac{1}{\mu} {\vec{B}} = \vec{J}
\end{equation}

Jelikož cílem odvozování matematického modelu problému je převedení veličin charakteristických pro elektromagnetické pole na potenciály, v tomto případě na vektorový magnetický potenciál, byl použit vztah mezi \vec{B} a \vec{A}(\ref{magneticky_vektorovy_potencial}), tím se rovnice upraví na tvar:

\begin{equation}
 	\rot (\frac{1}{\mu} {rot \vec{A}}) = \vec{J}
 	\label{prvni_max_odvoz}
\end{equation}

Do tohoto momentu byl odvozován matematický model z 1. Maxwellovy rovnice, dále bude odvozována část z 2. Maxwellovy rovnice.
	
	Faradayův indukční zákon neboli druhá Maxwellova rovnice v diferenciálním tvaru ve znění:

\begin{equation}
	\rot \vec{E} = - \frac{\partial \vec{B}}{\partial t}
	\label{faradayuv_zakon}
\end{equation}

kde \vec{E} je intenzita elektrického pole. Opět je cílem převést z veličin charakteristických pro elektromagnetické pole na vektorový magnetický potenciál, proto je opět použit vztah (\ref{magneticky_vektorovy_potencial}) a tímto dojde ke změně tvaru rovnice:

\begin{equation}
	\rot \vec{E} = - \frac{\partial\rot\vec{A}}{\partial t}
	\label{faradayuv_zakon_potencial}
\end{equation}

Ten lze upravit na tvar:

\begin{equation}
	\rot \vec{E} = - \rot \frac{\partial\vec{A}}{\partial t}
	\label{faradayuv_zakon_potencial_rot}
\end{equation}

Odstraněním operátoru rotace vznikne konečný tvar rovnice pro intenzitu elektrického pole:

\begin{equation}
	\vec{E} = - \frac{\partial \vec{A}}{\partial t} - \grad{\varphi}
	\label{faradayuv_zakon_potencial_bez_rot}
\end{equation}

Dosazením ze konstitutivního vztahu pro proudovou hustotu \vec{J} a intenzitu elektrického pole \vec{E} (\ref{gama}) do rovnice (\ref{faradayuv_zakon_potencial_bez_rot}), lze tuto rovnici zapsat:

\begin{equation}
	\vec{J} = - \gamma \frac{\partial \vec{A}}{\partial t} - \gamma \grad{\varphi}
	\label{faradayuv_zakon_potencial_konstitutivni}
\end{equation}

kde $\gamma$ je konduktivita. V této chvíli lze rovnice (\ref{prvni_max_odvoz}) a (\ref{faradayuv_zakon_potencial_konstitutivni}) spojit, vznikne jedna rovnice ve tvaru:

\begin{equation}
	\rot (\frac{1}{\mu} {rot \vec{A}}) = - \gamma \frac{\partial \vec{A}}{\partial t} - \gamma \grad{\varphi}
	\label{matematicky_model_neupraveny}
\end{equation}

Dále lze provést úprava podle vztahů (\ref{gama}) a (\ref{intenzita_potencial}):

\begin{equation}
	\rot (\frac{1}{\mu} {rot \vec{A}}) = - \gamma \frac{\partial \vec{A}}{\partial t} + \vec{J_{ext}}
	\label{matematicky_model_stacionarni}
\end{equation}

kde $\vec{J_{ext}}$ je proudová hustota externích proudů. Lze také říci, že:

\begin{equation}
	- \gamma \frac{\partial \vec{A}}{\partial t} = \vec{J_{eddy}}
	\label{proudova_hustota_virive_proudy}
\end{equation}

kde $\vec{J_{eddy}}$ je vektor proudové hustoty vířivých proudů. Rovnice (\ref{proudova_hustota_virive_proudy}) lze dosadit do vztahu (\ref{matematicky_model_stacionarni}) a vyjde rovnice ve tvaru:

\begin{equation}
	\rot (\frac{1}{\mu} {rot \vec{A}}) = \vec{J_{eddy}} + \vec{J_{ext}}
	\label{matematicky_model_stacionarni_complete}
\end{equation}

A jelikož v magnetickém harmonickém poli platí:

\begin{equation}
	\vec{J_{eddy}} = - \gamma \frac{\partial \vec{A}}{\partial t} = - \gamma j \omega \vecfaz{A}
	\label{virive_proudy}
\end{equation}

Z dosazení rovnice (\ref{virive_proudy}) do rovnice (\ref{matematicky_model_stacionarni_complete}) se získá vztah:

\begin{equation}
	\rot (\frac{1}{\mu} {rot \vecfaz{A}}) = - \gamma j \omega \vecfaz{A} + \vec{J_{ext}}
\end{equation}

Pouhou jednoduchou úpravou rovnice se získá konečný tvar matematického modelu magnetického pole řešeného problému:

\begin{equation}
	\rot (\frac{1}{\mu} {rot \vecfaz{A}}) + \gamma j \omega \vecfaz{A} = \vec{J_{ext}}
	\label{matematicky_model_magneticke_pole}
\end{equation}