\section{Model teplotního pole}
Teplo se v tomto problému šíří vedením, proto je pro popis teplotního pole použita rovnice pro vedení tepla neboli Fourier-Kirchhoffova rovnice. Tu lze zapsat ve tvaru:

\begin{equation}
	\div ( \lambda \grad T) = \rho c \frac{dT}{dt} - p_J 
	\label{fourier_kirchhoff} 
\end{equation} \cite{TZEP}

kde $T$ je teplota, $\lambda$ tepelná vodivost, $\rho$ měrná hmotnost, $c$ měrná tepelná kapacita a   $P_J$ měrné Jouleovy ztráty.

Jouleovy ztráty, jsou ztráty vzniklé tepelnými účinky vířivých proudů a jsou definovány pomocí Jouleova zákona, který je ve znění:

\begin{equation}
	p_{J}=\frac{||\vec{J}_{v}||^{2}}{\gamma}
	\label{joule_ztraty} 
\end{equation} 

V dalším kroku se rozloží úplná derivace T vyskytující se v rovnici(\ref{fourier_kirchhoff}) na parciální derivace podle jednotlivých složek souřadného systému, v tomto případě osově symetrického souřadného systému:

\begin{equation}
	p_{J}=\frac{||\vec{J}_{v}||^{2}}{\gamma}
	\label{joule_ztraty} 
\end{equation}