\newpage

\section{Model teplotního pole}
Teplo se v tomto problému šíří vedením, proto je pro popis teplotního pole použita rovnice pro vedení tepla neboli Fourier-Kirchhoffova rovnice. Tu lze zapsat ve tvaru:

\begin{equation}
	\div ( \lambda  \grad T) = \rho c \frac{dT}{dt} - p_J 
	\label{fourier_kirchhoff} 
\end{equation} \cite{TZEP}

kde $T$ je teplota, $\lambda$ tepelná vodivost, $\rho$ měrná hmotnost, $c$ měrná tepelná kapacita a   $P_J$ měrné Jouleovy ztráty.

Jouleovy ztráty, jsou ztráty vzniklé tepelnými účinky vířivých proudů a jsou definovány pomocí Jouleova zákona, který je ve znění:

\begin{equation}
	p_{J}=\frac{||\vec{J}_{v}||^{2}}{\gamma}
	\label{joule_ztraty} 
\end{equation} 

V dalším kroku se rozloží úplná derivace T vyskytující se v rovnici(\ref{fourier_kirchhoff}) na parciální derivace podle jednotlivých složek souřadného systému, v tomto případě válcového souřadného systému:

\begin{equation}
	\frac{dT}{dt} =\frac{\partial T}{\partial t} + \frac{\partial T}{\partial r} \frac{d r}{d t} + \frac{\partial T}{\partial \alpha} \frac{d \alpha}{d t} + \frac{\partial T}{\partial z} \frac{d z}{d t}
	\label{rozklad_parcialni_derivace} 
\end{equation}

Časové derivace složek polohy lze podle znalostí z kinematiky nahradit složkami vektoru rychlosti, tedy $v_r$, $v_\alpha$ a $v_z$. Z této znalosti, z rovnice (\ref{joule_ztraty}) a (\ref{rozklad_parcialni_derivace}) vyplývá, že upravená rovnic (\ref{fourier_kirchhoff}) bude mít takovýto tvar:

\begin{equation}
	\div ( \lambda  \grad T) = \rho c (\frac{\partial T}{\partial t} + \frac{\partial T}{\partial r} v_r + \frac{\partial T}{\partial \alpha} v_\alpha + \frac{\partial T}{\partial z} v_z) - \frac{||\vec{J}_{v}||^{2}}{\gamma}
	\label{fourier_kirchhoff_upraven} 
\end{equation} 

Dále lze upravit výraz zahrnující parciální derivace teploty T:

\begin{equation}
    \frac{\partial T}{\partial t} + \frac{\partial T}{\partial r} v_r + \frac{\partial T}{\partial \alpha} v_\alpha + \frac{\partial T}{\partial z} v_z = \vec{v} \cdot \grad T
	\label{gradient_teplota} 
\end{equation} 

Toto lze dosadit do rovnice (\ref{fourier_kirchhoff_upraven}):
\begin{equation}
	\div ( \lambda  \grad T) = \rho c (\frac{\partial T}{\partial t} + \vec{v} \cdot \grad T ) - \frac{||\vec{J}_{v}||^{2}}{\gamma}
	\label{matematicky_model_teplotni_pole_rychlost} 
\end{equation} 

Dále lze vypustit člen s rychlostí, jelikož se nejedná o dynamický problém, ale pouze o statický problém, složky rychlosti a tedy i rychlost jako vektor je nulová. Po této úvaze lze zapsat rovnici ve tvaru:

\begin{equation}
	\div ( \lambda \grad T) = \rho c \frac{\partial T}{\partial t} - \frac{||\vec{J}_{v}||^{2}}{\gamma}
	\label{matematicky_model_teplotni_pole} 
\end{equation}

Tato rovnice reprezentuje matematický model teplotního pole.
